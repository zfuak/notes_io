\chapter{Two-sided market: Platform}
\section{Introduction}
\paragraph{Platform} Platforms organize/ facilitate exchanges between
agents. An important feature is the externality from the actions of each side.
\begin{remark}
    Now, not only the overall matters for the revenue, but also the price structure. Simply, how much and how the platform charges/subsidize each side.
\end{remark}
\begin{table}[!h]
	\centering
	\begin{tabular}{|c|c|c|}
        \hline
		Buyers           & Platform                      & Sellers         \\ \hline
		Gamers           & Videogame platforms           & Game developers \\ \hline
		Users            & Operating systems, App Stores &                 
		Application developers\\ \hline
		Viewers, readers & Portals, newspapers,TVs       & Advertisers     \\ \hline
		Cardholders      & Debit and credit cards        & Merchants       \\ \hline
	\end{tabular}
	\caption{Examples of two-sided platforms}
	\label{tab:two_sided_market}
\end{table}
\begin{remark}
    Platform must get both sides on board -- it may subsidize one side while making money on the other side
\end{remark}

\paragraph{Not every market is two-sided.} 
\begin{enumerate}
    \item purely vertical (Component supplier - manufacturer - customer): there's no externality from price structure.
    \item if direct negotiations are more effective (Bilateral electricity trading): A pair of generator and buyer should take into account only the total fee when bargaining for a bilateral energy trade. More specifically, they maximize $u(q)-c(q)-p_sq-p_b$ where $p$ is the variable fee charged by the transmission company. Therefore, what matters is the sum $p_s+p_b$ not the decomposition.
\end{enumerate}

\paragraph{Externality}
\begin{itemize}
    \item Usage externality : increased usage by one side raises the utility of the other side 
    \item Membership externality: increased membership on one side raises the potential for usage and thus the value of membership on the other side
\end{itemize}

\section{The club model}
%%% insert an image here
\subsection{Framework}
We define the utility of both sides (buyer \& seller), where $\beta$ represents externality effect.
\begin{equation*}
\begin{cases}
    u_B=\beta_B N_S-p_B\\
    u_S=\beta_S N_B-p_S\\
\end{cases}
\end{equation*}
The utility is translated into the number of users by the network size described below.
\begin{equation*}
    \begin{cases}
        N_B=D_B(u_B)\\
        N_s=D_S(u_S)\\
    \end{cases}
\end{equation*}
Therefore, price can be expressed as 
\begin{equation*}
    \begin{cases}
        p_B=\beta_B D_S(u_S)-u_B\\
        p_S=\beta_S D_B(u_B)-u_S\\
    \end{cases}
\end{equation*}
\paragraph{Pricing}
\begin{remark}
Given the simple framework, the intuition is that the \textbf{pricing rules must be adapted to the presence of cross externalities between groups}. Attracting a customer on one side of the market allows the platform to raise the price on the other side of the market. The platform can thus \textbf{sacrifice profit on one side to increase its profit of the other side.}  The stronger the externality generated by a side, the lower is the price charged on this side.
\end{remark}
Suppose the platform sells to one additional buyer, the induced value on the other side is $\beta_SN_S$. The monopoly can charge the sellers by this amount, keeping the sellers' participation constant.\\The platform's problem is
\begin{equation*}
    \begin{array}{ll}
        \text{max} &  \pi(u_B,u_S)\\&=(p_B-c_B)D_B(u_B)+(p_S-c_S)D_S(u_S)\\
         &=(\beta_B D_S(u_S)-u_B-c_B)D_B(u_B)+(\beta_S D_B(u_B)-u_S-c_S)D_S(u_S)
    \end{array}
\end{equation*}
The FOC w.r.t $u_B$ is 
\begin{equation*}
    (p_B-c_B)\frac{\partial D_B}{\partial u_B}-D_B+\beta_S\frac{\partial D_B}{\partial u_B}N_S=0
\end{equation*}
Alternatively we can write the problem as

\begin{equation*}
    \begin{array}{ll}
        \max_{p_B,p_S} & (p_B-c_B)N_B+(p_S-c_S)N_S\\
         &=(p_B-c_B)D_B(\beta_B N_S-p_B)+(\beta_S D_B(\beta_B N_S-p_B)-c_S)N_S
    \end{array}
\end{equation*}
The second term is not affected by $p_B$. Therefore the FOC is \begin{equation*}
    D_B(u_B)-(p_B-c_B)\frac{\partial D_B}{\partial u_B}-\beta_S N_S\frac{\partial D_B}{\partial u_B}=0
\end{equation*}
Rearrange, we get 
\begin{align*}
    p_B-c_B+\beta_S N_S&= D_B(u_B)/\frac{\partial D_B}{\partial u_B}\\
    \frac{p_B-c_B+\beta_S N_S}{p_B}&= \frac{1}{\eta_B}\\
\end{align*}
where $\eta$ is the elasticity of demand.\\
An \emph{intuitive} way to think about it is that 
\begin{align*}
    \text{Price} &= \text{opportunity cost} + \text{markup}\\
    &=\text{cost}-\text{value created on the other side}+ \text{markup}.
\end{align*}
Thus, it is natural that $p_B=c_B-\beta_S N_S+\frac{1}{\eta_B}p_B$.
\subsection{Welfare analysis}
Participants face heterogeneous opportunity cost of registering.
The utility is 
\begin{equation*}
    \begin{cases}
        \text{buyers: } u_B=\beta_B N_S-p_B+\alpha_{bi}\\
        \text{seller: } u_S=\beta_S N_B-p_S+\alpha_{si}\\
    \end{cases}
\end{equation*}
The $\alpha_{bi}$ is heterogeneous across agents, with density $f_B(\alpha)=-D'(\alpha)$.
\paragraph{Buyer surplus} 
The relationship between price and quantity is \begin{equation*}
    p_B=D_B^{-1}(N_B)+\beta_B N_S
\end{equation*}
insert a graph here.
Therefore, the buyer surplus is \begin{equation*}
    CS_B=\int_0^{N_B} (D_B^{-1}(N_B)+\underbrace{\beta_B N_S}_\text{externality of price on \# buyer}-p_B) dn
\end{equation*}
Similarly for seller.
\paragraph{Total welfare} The sum of seller, buyer and platform.
\begin{equation*}
    W=(\beta_B+\beta_S) N_S N_B + \int_0^{N_B} D_B^{-1}(N_B) dn +\int_0^{N_S} D_S^{-1}(N_S) dn-c_B N_B-c_S N_S
\end{equation*}
We write it in terms of utility \begin{equation*}
    W(u_B,u_S)=\pi(u_B,u_S)+v_B(u_B)+v_S(u_S)
\end{equation*}
where $v_B(u_B)$ represents buyer surplus and satisfies $v_B'(u_B)=N_B$.\\
The total welfare maximizing price is \begin{equation*}
    \begin{cases}
        p_B=c_B-\beta_S N_S\\
        p_S=c_S-\beta_B N_B
    \end{cases}
\end{equation*}
\begin{remark}
    The first best “optimal” price is equal to the cost net of the value created for other members of the network so as to induce the consumer to internalize the network effect. \emph{Yet}, this implies prices are below costs.
\end{remark}
\paragraph{Discussion: Can competitive access on each side work?} Competitive access corresponds to the case where on each side a large number of access providers sell access to the platform. Competitive access leads to prices equal to marginal cost on both sides. Thus competitive access leads to an inefficient price
structure (this contrast with one-sided networks). 
\textit{May monopoly access dominate competitive access?} Yes, if the monopoly can subsidize each side.
\paragraph{Discussion: Two-sided vs one-sided market} Consider a one-sided market with network effects: a software. There are two kinds of customers: firms and individuals. 
\begin{itemize}
    \item The utility of a firm: $\tau_A+\beta_A(n_A+n_B)$
    \item The utility of an individual: $\tau_B+\beta_B(n_A+n_B)$
\end{itemize}
Suppose that the company selling the software can
price-discriminate them. This setting is similar to the two-sided market: the main difference is that there are intragroup network effects in addition to the cross-group network effects. \textit{The key is whether we can price discriminate different groups based on their externalities.}
\section{The usage model}
Usage fee affects the probability of trade and the net benefits from trade.
\subsection{Framework}
insert a graph here. 
Assume there's no registration fee. The network size is \begin{equation*}
    \begin{cases}
        N_B=D_B(t_B)\\
        N_s=D_S(t_S)\\
    \end{cases}
\end{equation*}
Volume of transaction $D_B(t_B)D_S(t_S)$.\footnote{The demand assumes one transaction per potential pair.}
\subsection{Pricing}
Decompose the problem into two parts.
\begin{enumerate}
    \item First for a given total price, choose the price structure to maximize the volume of transactions \begin{equation*}
        \begin{array}{rl}
            V(t)=\max & D_B(t_B)D_S(t_S)\\
             \text{subject to}& t=t_B+t_S\\
        \end{array}
    \end{equation*}
    \item Balance the fees according to elasticity \begin{equation*}
        \frac{t_B}{t_S}=\frac{\eta_B}{\eta_S}.
    \end{equation*}
    \item Then choose the price level $\max_t (t-c)V(t)=(t_B+t_S-c) D_B(t_B)D_S(t_S)$. We fix the transaction fee on the sellers’ side and maximize profit on buyers. This is equivalent to the price formula in the club model because \begin{equation*}
        \underbrace{(t_B-(c-t_S))}_\text{profit net of opportunity cost} D_B(t_B)D_S(t_S).
    \end{equation*} Therefore \begin{equation*}
        \frac{t_B-c-t_S}{t_B}= \frac{1}{\eta_B}
    \end{equation*}
\end{enumerate}
\begin{remark}
    This leads to an efficient price structure but excessive price level.
\end{remark}
\section{Coordination failure}
\subsection{Background}
So far, we assume a unique consumer allocation for given prices. However there may be multiple allocations of consumers which may restrain the market power. More specifically, consumers may coordinate on the ``wrong'' allocations. There exhibits a chicken-and-egg problem : when starting a platform activity, a firm must convince some users to join while there is no counterpart yet! This may force the firm to: \begin{itemize}
    \item reduce some prices
    \item choose inefficient price structure
\end{itemize}
Suppose there are two homogenous sides of size 1, each with no intrinsic value and network externality $\beta_s$ and $\beta_b$. For any pair of positive prices less than the externality, there are two equilibrium allocations of consumers in the subgame 
\begin{itemize}
    \item Consumer optimistic beliefs: all consumers join
    \item Consumer pessimistic beliefs: no consumer joins.
\end{itemize}
The allocation where all consumers joins dominates the alloaction where none join. In the latter case there is a coordination failure.
Facing pessimistic beliefs, the platform has no choice than to set one price (slightly) below zero and then to charge a positive price $\beta_j$ for the other side : if $\beta_s+\beta_b>2c>\max \brac{\beta_s,\beta_b}$ the platform cannot operate!
\subsection{Business model}
\paragraph{Coordination issues} New platforms need to bring at
least one side “on board” to start the activity. Platforms have an interest in inducing efficient usage, insofar that they can capture the efficiency gains. For example, \begin{itemize}
    \item Complex tariffs may help for both aspects (chicken-and-egg and usage) but they may be a tension between objectives,
    \item Tying and bundling may help coordinate consumers,
    \item Vertical integration also helps by securing some participation on one side.
\end{itemize}
