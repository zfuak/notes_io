\chapter{Vertical relation: Upstream (M) and Dowstream (R)}
\section{Introduction}
\paragraph{Vertical relation}
Relationship between upstream and downstream parties: part supplier and manufacturer, manufacturer and retailer, franchiser and franchisee \emph{etc.} Whether a party is upstream or downstream depends on whether it is \emph{closer} to the consumer. For example: 
\begin{itemize}
    \item Brick-and-mortar store: supplier are upstream
    \item Online platforms: supplier are downstream
\end{itemize}
\paragraph{Vertical restraint}
Agreements between upstream and downstream parties: supply contracts, distribution contracts, licensing. Agreement/contract can serve different purposes. For example
\begin{itemize}
    \item Revenue sharing/pricing: linear pricing, \emph{non-linear} tariffs (\emph{e.g.} two part tariff). They can have a variety of names.
    \begin{itemize}
        \item Franchise fees
        \item Quantity rebates
        \item Fidelity rebates
        \item Slotting fees: manufacturer pays the retailer first.
        \item Menus of pricing schemes
    \end{itemize}
    \item Obligations/behavior restriction
    \begin{itemize}
        \item Service: M requires R provide x services to consumers.
        \item Quantity: quotas
        \item Price: Resale Price Maintenance (RPM), price ceiling/floor
        \item Bundling
        \item \textbf{Exclusivity}: exclusive territory, exclusive dealing
    \end{itemize}
    \item Integration/Merge and Acquisition: \emph{vertical} merge rather than horizontal.
\end{itemize}
\subsection{Antitrust Law}
\paragraph{Sherman Act}
\begin{enumerate}
    \item Section 1: prohibits contracts (or conspiracies) that restrain trade or commerce
    \item Section 2: prohibits monopolization or attempts to monopolize
\end{enumerate}
\paragraph{Treaty of Rome}
\begin{enumerate}
    \item Article 101: restricting competition is forbidden
    \item Article 102: prohibits abuses of a dominant position
    \begin{itemize}
        \item exploitative abuses: unfair terms (excessive prices, discrim.)
        \item exclusionary abuses: distorting competition (vert. foreclosure)
    \end{itemize}
\end{enumerate}

\section{Vertical coordination}
In this section, we will look at two coordination problems between M and R.
\subsection{Double marginalization}
\paragraph{Modeling choice}



\paragraph{Solution concept and algorithm} (Subgame perfect) Nash equilibrium using backward induction.

\paragraph{Result} R sets retail price p higher than the joint profit maximizing (\textit{jointly optimal}) retail price $p^M$.

\paragraph{Policy} There are several solutions to the problem. Each has different instrument, for example, margin, price, quantity.
\begin{enumerate}
    \item Eliminate one \textbf{margin}: \emph{e.g.} introduce more R (remove downstream margin), M will set the jointly optimal price.)
    \item Forcibly set $p^*$ as the \textbf{retail price} ceiling \footnote{The issue is that it requires M to have sufficient knowledge of the retail market.}
    \item Contracting on \textbf{retail quantity}: impose a minimum sales target on R.
    \item Align \textbf{incentive}: implement two-part tariff $\pa{w^*=c, F^*}$ \footnote{The fixed fee is used to share the profits.}
\end{enumerate}

\paragraph{Implication} The industry has incentive to remove double marginalization so as to maximize joint profit. The outcome is also good for the consumer. In this case, authority should \textit{laissez-faire}.

\subsection{Retail effort issue}
\paragraph{Modeling choice}
\paragraph{Result} Not only does R sets retail price $p>p^M$, but also it provides effort $e<e^M$.
\paragraph{Policy} Notice that on the demand side, \textbf{quantity} depends on \textbf{price} and \textbf{effort}. 
\begin{enumerate}
    \item Contract on two out of the \textbf{three dimensions}.
    \item Align \textbf{incentive} retailer becomes the residual (\textit{total profit - fixed fee}) claimant.\footnote{Increase the number of targets doesn't necessarily require additional instruments.}
\end{enumerate}
\paragraph{Question} Can we also introduce intraband competition as in double marginalization? Answer: it doesn't generate the jointly optimal $p^M$ and $e^M$. The effect can be analysed on two extreme cases, depending on whether other Rs can freeride one retailer's effort.
\begin{itemize}
    \item Free riding: effort is lower than $e^M$
    \item No free riding: effort can be either too low or too high. 
\end{itemize}
\subsection{Discussion}


%%%
\section{Horizontal coordination}

\subsection{Upstream coordination}
\paragraph{Modeling}
There are $M_A$, $M_B$ selling differentiated product to retailer $R_1$.
The first best result is such that as if $M_A$ and $M_B$ are competing directly on the market. \footnote{For example, the simplest Bertrand competition that we learn from undergrad econ.} Here, we analyse two cases. The case where seller $M$ has the bargaining power/makes the first move is more interesting. 
\paragraph{Result} Similar as before, double marginalization.
\paragraph{Solution} Following from the discussion in previous sections, in order to achieve industry-wide optimal outcome, we would need two-part tariff to align the incentive. 
\paragraph{Discussion} \label{upstr_coor:discussion} With incentive aligned, there are three potential outcomes.
\begin{itemize}
    \item $\pa{M_A, R}$ exclusive dealing:
    \item $\pa{M_B, R}$ exclusive dealing:
    \item $\pa{M_A, M_A, R}$ common agency: 
\end{itemize}
\begin{remark}
    In the absence of agency problems, there always exists a (more efficient) equilibrium with common agency.
\end{remark}
\begin{remark}
    There's always ``\textit{agency rent}'' left to the retailer. Manufacturers can not appropriate all industry-wide profit.
\end{remark}
\begin{remark}
    Public or secrete contracting doesn't matter.
\end{remark}

\subsection{Downstream coordination}
There are one upstream $M$ and two downstream firms $R_1,R_2$. Let us suppose that $M$ has the bargaining power. 
\begin{remark}
    There exhibits differences between public contracting and secrete contracting.
\end{remark}
\subsection*{Public contract}
\paragraph{Modelling: M power}
$M$ wants to set $\pa{w_1,w_2}$ such that the market outcome is the monopoly differentiated/identical one.
\paragraph{Result} If $M$ charge at costs as solving the double marginalization problem, then due to downstream competition, the price will be lower than monopoly price.
\paragraph{Policy} Let $M$ charge \textbf{above costs}, less or equal to \textbf{below monopoly price}.

\paragraph{Modelling: R power}
Retailers offer tariff to $M$. 
\paragraph{Policy} We introduce \textit{three-part tariff}--slotting fee. Each $R_i$ charges $M$ a certain amount of slotting fee $S_i$ and set $w_i=c_i$. If $R_i$ is the only retailer in the market, the industry $\pa{M,R_i}$ total profit is $\pi^M$. So it promised $M$ that $F_i=\pi^M$ will be given back as long as it is unique. If the $M$ agrees, then in the end $R_i$ gets $S_i$ and Manufacturer gets $\pi^M-S_i$.
\begin{remark}
    This is \emph{de facto} exclusive dealing.
\end{remark}
Since the two retailer produces differentiated product, they can have different monopoly profit $\pi_i^M$. The more efficient retailer with higher $\pi_i^M$ wins the exclusive deal. 
\paragraph{Discussion}
Unlike the multiple equilibria discussed in \ref{upstr_coor:discussion}, with the slotting fee introduced above, only exclusive dealing equilibrium exists. Because both $M$ and $R$ could benefit from exclusive dealing.
In any candidate "common agency" equilibrium, $M$ must be indifferent between accepting both or either offer. Otherwsise the favoured retailer could ask for better terms. Speicifically, we should have
\begin{equation*}
    \pi_1+\pi_2-S_1-S_2\geq \pi_1^M-S_1^M
\end{equation*} where $\pi_i$ is the monopoly profit of $R_i$ and $S_i$ is the slotting fee under common agency. The RHS is counterpart under exclusive dealing.
Therefore, we should have $\pi_2-S_2 =\pi_1^M-S_1^M-\pi_1+S_1$ (constraint should bind). Similarly, we have $\pi_1-S_1 =\pi_2^M-S_2^M-\pi_2+S_2$. Adding these two together, \begin{equation*}
    \pi_1+\pi_2-S_1-S_2=\frac{1}{2}\pi_1^M-S_1^M+\pi_2^M-S_2^M<\max\set{\pi_1^M-S_1^M,\pi_2^M-S_2^M}
\end{equation*}
Therefore, the common agency equilibrium is not sustainable. $M$ and the more efficient $R_i$ would jointly benefit from exclusivity.

\subsection*{Secret contract}
\paragraph{Modeling: M power} 
There is \textbf{upstream opportunism} when $R_i$, $R_j$ do not observe the contract received by the other. When $M$ is contracting with one retailer, it can free-rides on other retailer’s sales.
\paragraph{Result} We discuss the outcome under Cournot and Bertrand competition.
\begin{remark}
    Depending on the beliefs of the retailer (\textit{``How to interpret deviant offers from $M$''}), there exists different equilibrium.
\end{remark}
\begin{itemize}
    \item Cournot competition
    \begin{itemize}
        \item Passive belief: perfect competition outcome
        \item Symmetric belief: monopoly outcome
    \end{itemize}
    \item Bertrand competition
    \begin{itemize}
        \item Passive belief: Direct Bertrand competition
    \end{itemize}
\end{itemize}
\paragraph{Policy}
\begin{itemize}
    \item Reputation
    \item Exclusive dealing
    \item RPM: 
        \begin{quote}
            industry-wide price floor\\
            bilateral price ceilings to squeeze downstream margin
        \end{quote}
\end{itemize}

\paragraph{Implication}
There are conflicting interests between firms’ profits and consumer surplus or total welfare. There's ground for intervention.
%%% 
\section{Competing vertical structures}
\subsection{Strategic delegation/Competition among vertical structures}
\paragraph{Modeling}
insert a picture.
There are two $M_A, M_B$. $M_A$ deals with $R_1^A, R_2^A$. Retailer only resells, therefore each retailer sells exactly the same product. Due perfect competition, retailers earn zero margin. For M, setting whole sale $w_i^j$ amounts to setting retail price. It is as if brand A and B are directly competing with each other.
We regard this scenario as the \textbf{benchmark}.
\paragraph{Policy} Now $M_A$ deals exclusively with $R_1^A$.
\paragraph{Implication} The implementation of exclusive territory dampens competition (strategic delegation). It reduces intraband competition. Retail prices respond to rival’s wholesale prices, whih gives rise to higher wholesale (and retail) prices.
\begin{remark}
    clear conflict between private/social interests
\end{remark}
\paragraph{Discussion}
Issue of observability and credibility. 
Credibility differs in types of competition.
\begin{quote}
    Bertrand competition: strategic complementarity\\
    Cournot competition: strategic substitute.
\end{quote}
    
\subsection{Facilitating practice}
\paragraph{Background} There exists different types of coordination/collusion. 
\begin{itemize}
    \item Tacit coordination
    \item Explicit coordination: communication, meetings, down raids, leniency programs
    \item Tacit coordination with explicit ``\textbf{facilitating practices}'': information sharing, trade associations, marketing practices (``price insurance''), multi-market contact, \textbf{vertical restraints} \emph{etc.}
    \begin{quote}
        \textbf{An interesting example is that by making the market more transparent, it can facilitate tacit collusion.} For example, the transportation authority releases vehicle registration record such that the vehicle manufacturer knows what others are producing without explicit communication. This help with tacit collusion. Another example is the gasoline market. The authority publishes the price of gasoline in different stations, trying to help consumer to find the cheapest gasoline. However, it also helps the gasoline station to tacitly collude.
    \end{quote} 
\end{itemize}
\paragraph{Modelling: Dynamic setting}
There are two M $\pa{M_A,M_B}$, each dealing with a single retailer $\pa{R^A,R^B}$. There's no intraband competition. In a static setting with deterministic demand, each adopting two part tariff will achieve Bertrand competition outcome. However, now 1) Manufacturers are long lived with some discount factor $\delta$. 2)There's demand shock to both products.
\paragraph{Policy} RPM as vertical restraint.
\begin{remark}
    Due to shock, price flexibility is desirable for the industry. There's a tradeoff between RPM (as collusion facilitating device) and price flexibility.
\end{remark}
\paragraph{Discussion}
\paragraph{Implication}
We know that RPM \emph{can} help firms collude and generate higher industry profit. We now analyze the impact of RPM on consumers. \\
When shocks are on demand: consumers like price rigidity. Whether RPM is beneficial to customers depends on the tradeoff between price rigidity and (high) price level.\\
When shock on cost: consumers prefer flexible prices. RPM reduces consumer welfare surely. 

\section{Multilateral relations}
\section{Exclusionary practices}


